\chapter{Introduction}


KNN (K Nearest Neighbour) Join is one of the simplest and elegant
classification or regression algorithm which works remarkably well in
practice. It is a non parametric lazy learning algorithm. Unlike eager
learning algorithms like SVM, decision trees, building a model is
cheap. It is
classified as one of the top 10 machine learning algorithm \cite{wu_top_2008}. KNN Join
has become an important primitive in datamining and
finds its application in multimedia data retrieval, Online
recommendation, DNA search,
Molecular Biology, bioinformatics  and
many other fields.

\medskip

One major drawback of KNN Join with regards to performance is that the computational complexity is
very high. KNN Join, if run on a single node, with a
large training and test dataset, does not provide results in reasonable amount of time, making it
unusable. This warrants the need for better algorithm with
reduced computational complexity and should be able to
leverage distributed frameworks like Hadoop \cite{hadoop_mr} or Spark
\cite{apache_spark} to run on datacenter scale machines in parallel.
Though there has been many research in KNN Join, they are mostly focussed on either creating
an efficient indexing techniques or efficient pruning logic in
partitioned datasets for reducing
number of distance computation in a centralized environment.
Only one
research paper \cite{lu_efficient_2012} has been published so far in
designing an accurate algorithm
for Distributed framework. They chose to use Hadoop MapReduce as it
was the most dominant framework at that time.


In this paper, we focus on
developing an accurate KNN Join algorithm for spark framework, which is
the current high performant distributed framework. Our goal in this
paper is to reduce the number of distance computations, to improve running time,
to be scalable from single node to thousands of node and to be scalable from
medium to large dataset.

This thesis is organized as follows. In chapter 2, we will give a
background of KNN Join, Spark and dwell upon some of the previous
research works. In chapter 3, we will talk in detail about our
solution to the KNN Join problem. And finally in chapter 4, we will analyse our
experimental results.
